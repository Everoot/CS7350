\documentclass[12pt]{article}
 
\newenvironment{sol}[1][Solution]{\begin{trivlist}\item[\hskip\labelsep {\bfseries #1:}]}{\end{trivlist}}
\usepackage{minted}
%\usemintedstyle{perldoc}
\usemintedstyle{vs}
\usepackage{graphicx}
\graphicspath{./}

\usepackage[margin=1in]{geometry} 
\usepackage{amsmath,amsthm,amssymb}
\usepackage{times,url}

\begin{document}
\renewcommand{\qedsymbol}{\filledbox}
\begin{center}
    \textbf{CS 5/7350} \\
    \textbf{Quiz \#1 Due Feb 22 for Completion Grade}
%replace X with the appropriate number
\end{center}
\begin{flushright}
\underline{Name \& ID: Bingying Liang \ 
48999397 \ \ \ \ \ \ \ \ \ \ \ \ \ \ \ CS5350? Yes / No $\surd$}
\end{flushright}

\begin{enumerate}
    \item \ [1 pt] Given that $(2^n) \% M = 33 $ and M is $> 100$. Given the value for $(2^{n+1})\% M $ \underline{\ \ 66 \ \ }
    \begin{sol}
        %\hspace*{\fill} \\
        \begin{align*}
        & \because (2^{n}) \% M = 33 \\
        & \therefore 2^{n + 1} \% M =  (2^n \times 2^1) \% M = (2^n \% M) \times (2^1 \% M) = 33 \times (2 \% M)\\
        & \because M > 100 \\
        & \therefore 2 \% M = 2 \\
        & \therefore 2^{n+1} \% M = 33 \times 2 = 66
        \end{align*}
    \end{sol}

    \item \ [2 pt] Consider the following function:
    \begin{minted}[framesep=2mm,baselinestretch=1.2,fontsize=\footnotesize]{text}
    #include <stdio.h>
    function (int n)
    {
        product = 1;
        for (i = 1; i <= n; i++){
            for (j = 1; j <= i; j++){
                product = product * j;
            }
        }
        printf("%d\n", proudct);
    }
\end{minted}
    \begin{enumerate}
        \item Write a function for the number of multiplications performed vs $n$.
        \begin{sol}
        \begin{align*}
            f(n) = (1 + 2 + 3 + \dots + n)  =  \frac{n(1+n)}{2} 
        \end{align*}
        \end{sol}
        
        \item What is the asymptotic running time of the code using "multiplication" as a basic element.
        
        \begin{sol}
        \begin{align*}
        & \lim_{n \rightarrow{\infty}}\frac{n(1+n)}{2} = \Theta(n^2)  
        \end{align*}
        \end{sol}
    \end{enumerate}


    \item \ [2 pts] A program can process 3000 item in 11 seconds.
    \begin{enumerate}
        \item About how long would it take to process 15000 items if the function describing the running time is bounded by $\Theta(n)$?
        \begin{sol}
        \begin{align*}
            &\because \Theta{(n)}, 3000 \ items \ in \ 11 \ seconds\\
            &\therefore \frac{15000}{3000} = 5 \\
            &\therefore time = 5 \times 11 = 55 \ seconds
        \end{align*}
        \end{sol}

        \item About how long would it take to process 15000 items if the function describing the running time is bounded by $\Theta(n^3)$?
        \begin{sol}
            \begin{align*}
            &\because \Theta{(n^3)}, 3000 \ items \ in \ 11 \ seconds\\
            &\therefore \frac{15000^3}{3000^3} = (\frac{15000}{3000})^3 = (5)^3 = 125 \\
            &\therefore time = 125 \times 11 = 1375 \ seconds
            \end{align*}
        \end{sol}

        \item About how long would it take to process 15000 items if the function describing the running time is bounded by $\Theta(2^n)$
        \begin{sol}
            \begin{align*}
            &\because \Theta{(2^n)}, 3000 \ items \ in \ 11 \ seconds\\
            &\therefore \frac{2^{15000}}{2^{3000}} = 2^{15000-3000}=2^{12000} \\
            &\therefore time = 2^{12000} \times 11 \ seconds
            \end{align*}
        \end{sol}
    \end{enumerate}
\end{enumerate}

\end{document}
