\documentclass[12pt]{article}
 
\newenvironment{sol}[1][Solution]{\begin{trivlist}\item[\hskip\labelsep {\bfseries #1:}]}{\end{trivlist}}
\usepackage{minted}
%\usemintedstyle{perldoc}
\usemintedstyle{vs}
\usepackage{graphicx}
\graphicspath{./}
\usepackage{diagbox}
\usepackage[margin=1in]{geometry} 
\usepackage{amsmath,amsthm,amssymb}

\usepackage{latexsym}
\usepackage{times,url}     % 字体
\usepackage{cprotect}
\usepackage{listings}

\usepackage{tikz-qtree}
\usepackage{forest}
 

\begin{document}
\renewcommand{\qedsymbol}{\filledbox}
\begin{center}
    \textbf{CS 5/7350} \\
    \textbf{Quiz \#2 Due Mar 1 for Completion Grade}
%replace X with the appropriate number
\end{center}
\begin{flushright}
\underline{Name \& ID: Bingying Liang \ 
48999397 \ \ \ \ \ \ \ \ \ \ \ \ \ \ \ CS5350? Yes / No $\surd$}
\end{flushright}
\begin{enumerate}
    \item \ [1 pt] Argue that the problem, S, of sorting an unsorted array of integers of length greater than 100 elements is at least as hard - and maybe even harder - than the problem, L, of finding the ten largest elements of the same unsorted array of integers.
    \begin{sol}
    \hspace*{\fill}\\
    S is at least as hard and maybe even harder than the problem L. \\
    Because if solve the problem S, and then can use the solution of S to solve the problem L which means the solution of problem L is based on the solution of S.
    \end{sol}

    \item \ [2 pts] A tree has the following In-Order and Pre-Order traversals. Draw the tree
    \begin{align*}
        \text{In Order: A X R M B L H P Z S Q}\\
        \text{Pre Order: M R A X P L B H Q Z S}
    \end{align*}
    \begin{sol}
    \hspace*{\fill}\\
    \begin{align*}
    \begin{tikzpicture}[level/.style={sibling distance=45mm/#1}]
    \node{M}
    child { node{R} 
            child { node{A} 
                    child[missing]
                    child{ node{X}}
                  }
            child[missing]
          }
    child { node{P}
            child { node{L}
                    child{ node {B}}
                    child{ node {H}}
                    }
            child { node{Q}
                    child{ node{Z}
                           child[missing]
                           child{ node{S}}
                         }
                    child[missing]
                    }
        };
    \end{tikzpicture}
    \end{align*}

    \end{sol}
    
    \item \ [1 pts] Answer the following 3 questions:
    \begin{enumerate}
        \item How much entropy does an entire message with 40A's and 60 B's have?
        \begin{sol}
        \begin{align*}
            &A: p_A = \frac{40}{100} = \frac{2}{5}, \log_2{\frac{1}{p_A}} = \log_2{\frac{1}{\frac{2}{5}}} = \log_2{\frac{5}{2}} \ bits \\
            &B: p_B = \frac{60}{100} = \frac{3}{5}, \log_2{\frac{1}{p_B}} = \log_2{\frac{1}{\frac{3}{5}}} = \log_2{\frac{5}{3}} \ bits \\
            &Total = 40 \times \log_2(\frac{5}{2}) + 60 \times \log_2(\frac{5}{3}) \approx 97.095\  bits
        \end{align*}
        \end{sol}
        \item How much entropy does an entire message with 50A's and 50 B's have?
        \begin{sol}
        \begin{align*}
            &A: p_A = \frac{50}{100} = \frac{1}{2}, \log_2{\frac{1}{p_A}} = \log_2{\frac{1}{\frac{1}{2}}} = \log_2{2}=1 \ bits \\
            &B: p_B = \frac{50}{100} = \frac{1}{2}, \log_2{\frac{1}{p_B}} = \log_2{\frac{1}{\frac{1}{2}}} = \log_2{2}=1 \ bits \\
            &Total = 50 \times 1 + 50 \times 1 = 100 \ bits
        \end{align*}
        \end{sol}
    \end{enumerate}
    \item \ [2 pts] You have a complete graph with $|V|$ vertices where $|V|$ is $ \geq 2$. Each edge in this graph has a capacity of 7. You pick one vertex as the Start Vertex, S, and another vertex as the Sink Vertex, T. Since the is a complete graph, you will get the same answer regardless of which two vertices you pick. Answer the following questions:
    \begin{enumerate}
        \item What is the length of the shortest path between Vertex S and Vertex $T$
        \begin{sol}
        \hspace*{\fill}\\
        Edge of Vertex $S$ and Vertex $T$, i.e. $E(S, T)$.\\
        Because it is a complete graph, it must have an edge between vertex $V$ and Vertex $T$.
        \end{sol}
        \item What is the maximum flow (in terms of $|V|$) between Vertex S and Vertex T
        \begin{sol}
        \begin{align*}
            (|V|-1) \times 7
        \end{align*}
        \end{sol}
        \item What is the weight of the minimum spanning tree of the graph? 
        \begin{sol}
        \begin{align*}
            (|V| - 1) \times 7
        \end{align*}
        \end{sol}
    \end{enumerate}
\end{enumerate}
\end{document}
