\documentclass[12pt]{article}
 
\newenvironment{sol}[1][Solution]{\begin{trivlist}\item[\hskip\labelsep {\bfseries #1:}]}{\end{trivlist}}
\usepackage{minted}
%\usemintedstyle{perldoc}
\usemintedstyle{vs}
\usepackage{graphicx}
\graphicspath{./}

\usepackage[margin=1in]{geometry} 
\usepackage{amsmath,amsthm,amssymb}
\usepackage{times,url}
\usepackage{tikz}
\usepackage{color}
\usepackage{enumerate}
\begin{document}
\renewcommand{\qedsymbol}{\filledbox}
\begin{center}
    \textbf{CS 5/7350 - Test\#2} \\
    \textbf{April 21, 2021}
%replace X with the appropriate number
\end{center}
\begin{flushright}
Name: \underline{Bingying Liang }\\
ID:  \underline{\ \ \ \ \ 48999397 \ \ \ \ \ }
\end{flushright}
\begin{enumerate}
    \item \ [8 pts] Consider the following NP completeness questions. Answer them with “some” “all” “none” or “unknown”
    \begin{enumerate}
        \item Which Problems in NP are also in P? (“some” “all” “none” or “unknown”)
        \begin{sol}
            unknown
        \end{sol}
        \item Which problems in NP are also in NP-Hard? ( “some” “all” “none” or “unknown”)
        \begin{sol}
            $<$ Skip not cover yet $> $
        \end{sol}
        \item If someone can solve the Circuit Sat problem in Polynomial Time, then all NP and all NP complete problems can be solved in Polynomial Time? (True or False)
        \begin{sol}
            False
        \end{sol}
        \item NP means Non-Polynomial? (True or False)
        \begin{sol}
            False
        \end{sol}
        \item Some NP problems can be solved in polynomial time? (True or False)
        \begin{sol}
            True
        \end{sol}
        \item Which NP-Hard Problems are also NP-Complete? ( “some” “all” “none” or “unknown”)
        \begin{sol}
            some
        \end{sol}

    \end{enumerate}

    \item \ [8 pts] Argue that the Hamiltonian Cycle is NP-Complete given that the Hamiltonian Path is NP-Complete
    \begin{sol}
        $<$ Skip not cover yet $>$
    \end{sol}
    \item \ \textcolor{red}{[7 pts] Compute the value for Z given that ( (161 * Z) + 5729 ) modulo 11609 = 11169}
    \begin{sol}
    \begin{align*}
        & (161 \times Z + 5729) \bmod 11609 = 11169 \\
        & (161 \times Z) \bmod 11609 + 5729 \bmod 11609 = 11169 \\
        & (161 \times Z) \bmod 11609 + 5729 = 11169 \\
        & (161 \times Z) \bmod 11609 = 5540 \\ 
        & (161 \times Z) \bmod 11609 = 5540 \bmod 11609 \\ 
        & (\frac{1}{161}\times 161 \times Z) \bmod 11609 = (\frac{1}{161} \bmod 11609) \times (5540 \bmod 11609)\\
        & Z \bmod 11609 = (\frac{1}{161} \bmod 11609) \times (5540 \bmod 11609) \\
    \end{align*}
    \begin{center}
        \begin{tabular}{|c|c|c|c|c|c|c|}
        \hline
             & A & B & Q & R & $\alpha$ & $\beta$   \\
             \hline
        -1   &   &   &   &   &      1    &  0 \\
        \hline
        0    & 11609  &  161  & 72   & 17   &      0    &  1 \\
        \hline
        1    & 161   &  17  & 9 & 8 & 1 & -72 \\
        \hline
        2 & 17 & 8 & 2  & 1 & -9 & 649 \\
        \hline
        3 & 8 & 1 & 8 & 0 & 19 & -1370 \\
        \hline
        4 & 1 & 0 & - & - & -161 & 11609 \\
        
        \hline
        \end{tabular}
        \begin{align*}
            & 19 \times 11609 - 1370 \times 161 = 1\\
            & (19 \times 11609) \bmod 11609 - (1370 \times 161) \bmod 11609 =  1 \bmod 11609\\
            & (19 \times 11609-1370 \times 161) \bmod 11609 = 1  \\
             & (- 1370 \times 161) \bmod 11609 = 1\\
            & (\frac{1}{161}) \times 161 \bmod 11609 = 1 \\ 
            & \therefore (\frac{1}{161})\bmod 11609 = (-1370) \bmod 11609\\
            & (\frac{1}{161} \bmod 11609) = (11609 - 1370) \bmod 11609\\
            & (\frac{1}{161}) \bmod 11609 = 102309\\
            & \therefore Z \bmod 11609 = 102309 \times (5540 \bmod 11609) \\
            & Z \bmod 11609 = (102309 \times 5540) \bmod 11609\\
            & Z \bmod 11609 = 2486\\
            & \therefore Z = 11609i + 2486, \text{where i is an integer}.
        \end{align*}
    \end{center}
    \end{sol}

    \item \ [8 pts] How many colors are needed to color the following special graphs:
    \begin{enumerate}
        \item A complete graph with $|V|$ vertices.
        \begin{sol}
        $|V|$
        \end{sol}
        \item A cycle with an odd number of vertices
        \begin{sol}
        3
        \end{sol}
        \item A tree
        \begin{sol}
        2
        \end{sol}
        \item A bipartite graph with 8 vertices in one partition and 9 vertices in the other partition.
        \begin{sol}
        2
        \end{sol}
    \end{enumerate}

    \item \ [12 pts] You have 4 dice. Each one is different. Die \#1 has sides \{ -1, 0, 1 \}. Die \#2 has sides \{ -2, -2, 0, 0 \} Die \#3 has sides \{1, 1, 1, 1\} and Die \#4 has sides \{0,0,0, 2,2,2\}
    \begin{enumerate}
        \item Fill in the table below
        \begin{sol}
        \hspace*{\fill}
        \begin{center}
            \begin{tabular}{|c|c|c|c|c|}
            \hline
                 & Die\#1 & Die \#1,\#2 & Die \#1, \#2, \#3 & Die \#1,\#2,\#3,\#4  \\
                 \hline
                -3 & 0 &2 & 0& 0 \\
                \hline
                -2 & 0 &2 & 8& 24 \\
                \hline
                -1 & 1 &4 & 8& 24 \\
                \hline
                0& 1 &2 & 16& 72 \\
                \hline
                1& 1 &2 & 8& 48 \\
                \hline
                2& 0 &0 & 8& 72 \\
                \hline
                3& 0 &0 & 0& 24 \\
                \hline
                4 & 0 &0 & 0& 24 \\
                \hline
                5& 0 &0 & 0& 0 \\
                \hline
                Sum& 3 &12 & 48& 288 \\
                \hline
            \end{tabular}
        \end{center}
        \end{sol}
        \item How many ways can you roll a 0 with these 4 dice?
        \begin{sol}
        72
        \end{sol}
        \item What is the probability of rolling a 0 with these 4 dice?
        \begin{sol}
        $\frac{1}{4}$
        \end{sol}
        \item How many ways can you roll a 4 with these 4 dice?
        \begin{sol}
        24
        \end{sol}
        \item What is the probability of rolling a 4 with these 4 dice?
        \begin{sol}
        $\frac{1}{12}$
        \end{sol}
    \end{enumerate}

    \item \ [12 pts] In the standard definition of a longest increasing subsequence of integers, each value must be at least 1 greater than the one before it. Consider, now, a longest 2-increasing subsequence where each value must differ by at least 2 instead of 1.\\
    As an example, if the original sequence is 6,2,3,4,7 a regular longest increasing subsequence would be 2,3,4,7 but a longest 2-increasing subsequence would be 2,4,7.\\
    Create the table for the longest 2-increasing subsequence of the sequence below and give the sequence:
    \begin{align*}
        3, 9, 6, 7, 14, 8, 11, 17, 12, 13, 20, 16, 17, 18, 23, 20, 24
    \end{align*}

    The longest 2-incraseing subsequence is:
    \begin{sol}
    \hspace*{\fill}
    \begin{center}
        \begin{tabular}{|c|c|c|c|c|c|c|c|c|c|c|}
        \hline
             3& 3& & & & & & & & & \\
                     \hline
             9& 3& 9 & & & & & & & & \\
                     \hline
             6& 3& 6 & & & & & & & & \\
                     \hline
             7 & 3& 6& & & & & & & & \\
                     \hline
             14 & 3& 6& 14& & & & & & & \\
                     \hline
             8& 3& 6& 8 & & & & & & & \\
                     \hline
             11& 3&6 &8 & 11& & & & & & \\
                     \hline
             17 & 3& 6 & 8 &11 &17 & &  & & & \\
                     \hline
             12 & 3& 6 & 8& 11&17 & & & & & \\
                     \hline
             13&  3& 6&8 & 11& 13& & & & & \\
                     \hline
             20 & 3&6 &8 &11 &13 & 20& & & & \\
                     \hline
             16& 3&6 &8 &11 &13 &16 &&  & & \\
                     \hline
             17 & 3&6 &8 &11 & 13&16 & & & & \\
                     \hline
             18&  3&6 &8 &11 &13 & 16&18 & & & \\
                     \hline
             23 & 3&6 &8 &11 & 13&16 &18 &23 & & \\
                     \hline
             20&  3&6 &8 &11 & 13&16 &18 & 20& & \\
                     \hline
             24&  3&6 &8 &11 & 13&16 &18 &20 & 24 & \\
                     \hline
        \end{tabular}
    \end{center}
    The longest 2-incraseing subsequence is 3, 6, 8, 11, 13, 16, 18, 20, 24 
    \end{sol}

    \item \ [12 pts] You have received a message that was compressed with LZW. Remember that A=65, B=66, C=67, D=68 and E=69. The dictionary starts with entry 256. The message you received was
    \begin{align*}
        67 \ 65  \ 68 \ 65 \ 257 \ 256 \ 69 \ 258 \ 260 
    \end{align*}

    \begin{enumerate}
        \item What was the original message and what is your dictionary after decompression?
        \begin{sol}
        \hspace*{\fill}\\
        \begin{center}
        \begin{tabular}{|c|c|}
        \hline
             Dictionary  \\
             \hline 
             A = 65 \\
             \hline
             B = 66 \\
             \hline 
             C = 67 \\
             \hline 
             D = 68 \\
             \hline 
             E = 69 \\
             \hline 
             ...\\
             \hline
             256 = CA \\
             \hline 
             257 = AD \\
             \hline 
             258 = DA \\
             \hline 
             259 = AA \\
             \hline
             260 = ADC \\
             \hline 
             261 = CAE \\
             \hline 
             262 = ED \\ 
             \hline
             263 = DAA \\
             \hline
        \end{tabular}
        \\
        \begin{tabular}{|c|c|c|c|c|c|c|}
        \hline
            &start \ $w$ & read $k$ & entry & output & Dictionary add & next \ $w$  \\
            \hline
             0& - & 67 & C & C & &  C  \\
             \hline 
            1&  C & 65 & A & A & CA = 256 & A \\ 
            \hline 
            2&  A & 68 & D & D &AD = 257 & D \\
            \hline
             3&  D & 65 & A & A &DA = 258 & A \\ 
             \hline
            4& A & 257 & AD & AD &AA = 259 & AD \\
            \hline
             5& AD & 256 &CA & CA &ADC = 260 & CA \\ 
             \hline
            6& CA & 69 & E & E &  CAE = 261 & E \\
            \hline
             7&  E & 258 & DA &DA & ED = 262 & DA   \\ 
             \hline
            8& DA &  260 & ADC & ADC & DAA = 263 & ADC \\
            \hline
             9&  &  &  & & & \\ 
             \hline

        \end{tabular}
                \end{center}
        \end{sol}
        \item Assuming 8 bits per character, how many bits were in the uncompressed message?
        \begin{sol}
        112 bits.
        \begin{align*}
           &\text{Entry:C,A,D,A,AD,CA,E,DA,ADC}\\
        &14 \times 11 = 112 \ bits
        \end{align*}
        \end{sol}
        \item Assuming the last entry of your dictionary was 1023, how many bits were in the compressed message
        \begin{sol}
        90 bits.
        \begin{align*}
            &\log_2{(1023+1)} = \log_2{1024} = 10\\
            &10 \times 9 = 90 \ bits
        \end{align*}
        \end{sol}
    \end{enumerate}

    \item \ [12 pts] Consider an RSA encryption system that has a public key of 7433 for the value of e and 21353 for the value of the modulus n. With a quantum computer, you are able to factor the 21353 into the product of two primes: 131 x 163.
    \begin{enumerate}
        \item Using this information, set up the table for the GCD (Extended Euclidian Algorithm)
        \begin{sol}
        \hspace*{\fill} \\
        public key: (e,n) = (7433, 21353)\\
        private key: (d,n)
        \begin{align*}
            &d = \frac{1}{e}\bmod \Phi(n) = \frac{1}{7433} \bmod \Phi(21353)\\
            &\Phi(21353)= \Phi(131 \times 163)=(131-1)\times(163-1)=130 \times 162= 21060\\
            &\therefore d = \frac{1}{7433} \bmod 21060
        \end{align*}
            \begin{center}
                \begin{tabular}{|c|c|c|c|c|c|c|}
                \hline
                     & A & B&Q &R &$\alpha$ & $\beta$  \\ 
                     \hline
                                       -1 &  & & & & 1 & 0 \\ 
                     \hline
                                         & 21060 & 7433 & 2 & 6194 & 0 & 1 \\ 
                     \hline
                                        &  7433 & 6194& 1 & 1239  & 1& -2 \\ 
                     \hline
                                        &  6194&1239 &4 &1238 & -1&  3  \\
                                        
                     \hline
                                        & 1239 &1238 &1 &1 & 5& -14 \\ 
                     \hline
                                        & 1238 & 1 & 1238 &0 & -6& 17 \\ 
                     \hline
                                        & 1 & 0 &- & -& 7433 & -21060 \\ 
                     \hline
                \end{tabular}
            \end{center}
            \begin{align*}
                &-(6) \times(21060)  + 17 \times 7433 = 1\\
                &(-6\times 21060) \bmod 21060 +(17\times 7433)\bmod 21060 = 1\\
                & (17 \times 7433) \bmod 21060 = 1\\
                & \because (\frac{1}{7433} \times 7433) \bmod 21060 = 1 \\
                & \therefore d = \frac{1}{17}\bmod 21060 = 17 \\
            \end{align*}
        \end{sol}
        \item What is the private key?
        \begin{sol}
        (17, 21353)
        \end{sol}
        \item \textcolor{red}{If you wanted to sign a message of value 3, what is the cipher text? (Compute the number)}
        \begin{sol}
            \begin{align*}
                3^{17} \bmod 21353 = 18572 \\
            \end{align*}
        \end{sol}
    \end{enumerate}

    \item \ [12 pts] You are interested in purchasing the items listed below. You have 14 points you can use to purchase items and you plan to pay cash for the rest. Setup and fill in the entire dynamic programming table for the problem and indicate which items you would purchase with points to minimize the cash you would have to spend for the rest.
    \begin{align*}
        &\text{Item } 1: 3 \text{ points, \$} 12 \\
        &\text{Item } 2: 4 \text{ points, \$} 14 \\
        &\text{Item } 3: 7 \text{ points, \$} 18 \\
        &\text{Item } 4: 4 \text{ points, \$} 10 \\
        &\text{Item } 5: 2 \text{ points, \$} 7 \\
    \end{align*}
    Which items would you take:
    \begin{sol}
    \hspace*{\fill}\\
    \begin{center}

    \begin{tabular}{|c|c|c|c|c|c|}
    \hline
         Sum points / items & 1  &1, 2  &1, 2, 3 &1, 2, 3, 4 & 1, 2, 3, 4, 5 \\
         \hline
            2      & 0& 0& 0& 0& 7  \\
         \hline
            3      & 12& 12& 12& 12 & 12  \\
         \hline
            4     & 12& 14& 14& 14 &  14\\
         \hline
            5      & 12& 14& 14&  14& 14 \\
         \hline
            6      & 12& 14& 14&  14& 21 \\
         \hline
            7      & 12& 26& 26&26  &  26\\
         \hline
            8      & 12& 26& 26&  26&  26\\
         \hline
            9      & 12& 26& 26& 26 &  33\\
         \hline
            10      & 12& 26& 30& 30&  33\\
         \hline
            11      & 12& 26&32 & 36&  36\\
         \hline
            12      & 12& 26&32 & 36&  37\\
         \hline
            13      & 12& 26&32 & 36&  43\\
         \hline
            14      & 12& 26& 44& 44&  44\\
         \hline
    \end{tabular}
        
    \end{center}
    44; item1, item2, item3. 
    \end{sol}
    \item \ [9 pts] The Levensthein Edit Distance determines the edit distance between two strings when Addition, Deletion and Substitution are allowed. Consider a different edit distance where only Addition and Deletion are allowed and Substituion is not.\\
    Assume you have two strings: A and B. The $i^{th}$ character of A is Ai and the $j^{th}$ character of B is Bj.
    \begin{enumerate}
        \item When considering the The $i^{th}$  character of A and the $j^{th}$ character of B, what is the “formula” for you would use for determining the value placed in the table at location i,j when finding the standard Levensthein Edit Distance
        \begin{sol}
        \hspace*{\fill}
             \begin{minted}[frame=lines,framesep=2mm,baselinestretch=1.2,fontsize=\footnotesize,linenos]{c}
// Base case
if (i == 0){
    T[i, j] = T[0,j];
}
if (j == 0){
    T[i, j] = T[i,0];
}

if (Ai == Bj){
    T[i,j] = min{T[i-1,j]+1, T[i,j-1]+1, T[i-1,j-1]};
}else{
    T[i,j] = min{T[i-1,j]+1, T[i, j-1]+1, T[i-1, j-1]+1};
}
\end{minted}
        \end{sol}
        \item When considering the The  $i^{th}$ character of A and the $j^{th}$ character of B, what is the “formula” for you would use for determining the value placed in the table at location i,j when finding the modified Levensthein Edit Distance without substitution
                \begin{sol}
        \hspace*{\fill}
             \begin{minted}[frame=lines,framesep=2mm,baselinestretch=1.2,fontsize=\footnotesize,linenos]{c}
// Base case
if (i == 0){
    T[i, j] = T[0,j];
}
if (j == 0){
    T[i, j] = T[i,0];
}

if (Ai == Bj){
    T[i,j] = min{T[i-1,j]+1, T[i,j-1]+1, T[i-1,j-1]};
}else{
    T[i,j] = min{T[i-1,j]+1, T[i, j-1]+1, T[i-1, j-1]+2};
}
\end{minted}
        \end{sol}
        \item When considering the The $i^{th}$ character of A and the $j^{th}$ character of B, what is the “formula” for you would use for determining the value placed in the table at location i,j when finding the Longest Common Subsequence
                \begin{sol}
        \hspace*{\fill}
             \begin{minted}[frame=lines,framesep=2mm,baselinestretch=1.2,fontsize=\footnotesize,linenos]{c}
// Base case
if (i == 0){
     T[i, j] = T[0,j];
     }
if (j == 0){
     T[i, j] = T[i,0];
}

if (Ai == Bj){
    T[i,j] = T[i-1,j-1]+1};
}else{
    T[i,j] = max{T[i-1,j], T[i, j-1]};
}
\end{minted}
        \end{sol}
    \end{enumerate}

\end{enumerate}
\end{document}
